% -*- mode: latex; TeX-master: t; -*-

% \documentclass[notes=hide]{beamer}
% \documentclass[notes=only]{beamer}
\documentclass[
14pt,
aspectratio=169,
usenames,
dvipsnames,
% handout,
x11names]{beamer}
%\let\Tiny=\tiny

% Font selection
\overfullrule=5pt

\usepackage{etex}
\usepackage{pgfpages}

\usepackage{tabularx}
\usepackage{multicol}
% \usepackage[subfolder]{gnuplottex}

%% beamerscape dependencies
\usepackage[absolute,overlay]{textpos}
\setlength{\TPHorizModule}{\paperwidth}
\setlength{\TPVertModule}{\paperheight}
\textblockorigin{0mm}{0mm}

\definecolor{greenvm}{HTML}{AFD18E}
\definecolor{g}{HTML}{0E3B12}
\definecolor{v}{HTML}{003366}

\newcommand{\src}[1]{\scriptsize Figure Source: \textit{#1}}

%%%%%%%%%%%%%%%%%%%%%%%%%%%%%%%%%%%%%%%%%%%%%%%%%%%%%%%%%%%%%%%%%%%%%%%%%%%%%%%%
% Beamer template
\usetheme[
	bullet=none,             % Default option: circle
%	nobigpagenumber,         % No big circled page number on lower right
	noheadline,		 % No headline
%       nofootline,		 % No footline
	% grey,                    % For a neutral grey background color
%	watermark=BG_both,	 % png file for the watermark
]{Ithaca}
% \usetheme{metropolis}
% \usetheme{Boadilla}
%\usetheme{Berlin}
% \usetheme{Warsaw}
% % \usetheme{Dresden}
% %\usecolortheme{seagull}
% \usecolortheme{beaver}
% %\usecolortheme{seahorse}
%\usefonttheme{serif}
% \usefonttheme{structurebold}

%%%%%%%%%%%%%%%%%%%%%%%%%%%%%%%%%%%%%%%%%%%%%%%%%%%%%%%%%%%%%%%%%%%%%%%%%%%%%%%%
% Template
%%%%%%%%%%%%%%%%%%%%%%%%%%%%%%%%%%%%%%%%%%%%%%%%%%%%%%%%%%%%%%%%%%%%%%%%%%%%%%%%

\definecolor{mypurple}{HTML}{494949}
\definecolor{myyellow}{HTML}{FFCC33}
\definecolor{mygrey}{HTML}{999999}

\setbeamercolor{alerted text}{fg=myyellow}
\setbeamercolor{headline}{bg=footline.bg}
\setbeamercolor{block body alerted}{bg=normal text.bg!90!black}
\setbeamercolor{block body}{bg=normal text.bg!90!black}
\setbeamercolor{block body example}{bg=normal text.bg!90!black}
\setbeamercolor{block title alerted}{fg=myyellow}
\setbeamercolor{block title}{bg=blue}
\setbeamercolor{block title example}{use={normal text,example text},fg=example text.fg!75!normal text.fg,bg=normal text.bg!75!black}
\setbeamercolor{frametitle}{fg=white}
\setbeamercolor{item projected}{fg=white}
\setbeamercolor{normal text}{bg=mypurple,fg=white}
\setbeamercolor{section in sidebar}{fg=brown}
\setbeamercolor{section in sidebar shaded}{fg=grey}
\setbeamercolor{separation line}{}
\setbeamercolor{sidebar}{bg=red}
\setbeamercolor{sidebar}{parent=palette primary}
\setbeamercolor{structure}{bg=black, fg=white}
\setbeamercolor{subsection in sidebar}{fg=brown}
\setbeamercolor{subsection in sidebar shaded}{fg=grey}

% \setbeamercolor{palette primary}{fg=mypurple!80!black}
% \setbeamercolor{palette secondary}{fg=mypurple!80!black}
% \setbeamercolor{palette quaternary}{fg=mypurple!50!black}
% \setbeamercolor{page number in foot}{fg=white}

% lets add page numbers
% \expandafter\def\expandafter\insertshorttitle\expandafter{%
%   \insertshorttitle\hfill\insertframenumber\,/\,\inserttotalframenumber}

% \setbeamercovered{transparent}
\beamertemplatetransparentcoveredhigh

% *****************************************
% >>>> 15.3.2    Using Beamer's Templates
% *****************************************
% As a user of the beamer class you typically do not 'use' or 'invoke'
% templates yourself, directly. For example, the frame title template
% is automatically invoked by beamer somewhere deep inside the frame
% typesetting process. The same is true of most other
% templates. However, if, for whatever reason, you wish to invoke a
% template yourself, you can use the following command.
% \usebeamertemplate***{ element name }
% -------------------------------
%%% 7.2.1 The Headline and Footline
% \setbeamertemplate{headline} % Beamer-Template/-Color/-Font
% \setbeamertemplate{headline}
% {%
%   \begin{beamercolorbox}{section in head/foot}
%     \vskip2pt\insertnavigation{\paperwidth}\vskip2pt
%   \end{beamercolorbox}%
% }
% \setbeamertemplate{headline}[default] % The default is just an empty headline.
% \setbeamertemplate{headline}[infolines theme]
% \setbeamertemplate{headline}[miniframes theme]
% \setbeamertemplate{headline}[sidebar theme]
% \setbeamertemplate{headline}[smoothtree theme]
% \setbeamertemplate{headline}[smoothbars theme]
% \setbeamertemplate{headline}[tree]
% \setbeamertemplate{headline}[split theme]
% \setbeamertemplate{headline}[text line]{ text } % The headline is typeset with 'text'
% Logo in every slide except title
% \addtobeamertemplate{headline}{}
% {%
% \vspace{1ex}
% \hspace{1ex}
% \includegraphics[height=1cm]{myrmigki_color}
% }
% -------------------------------
% \setbeamertemplate{footline} % Beamer-Template/-Color/-Font
% \setbeamertemplate{footline}[default]
% \setbeamertemplate{footline}[infolines theme]
% \setbeamertemplate{footline}[miniframes theme]
% \setbeamertemplate{footline}[page number]
% \setbeamertemplate{footline}[frame number]
% \setbeamertemplate{footline}[split]
% \setbeamertemplate{footline}[text line]{ text }
% % Footline in every slideexcept title
\setbeamertemplate{footline}{
  \leavevmode%
  \begin{beamercolorbox}[left,wd=.4\paperwidth,ht=2.5ex,dp=1.125ex,leftskip=.3cm, rightskip=.3cm plus1fil]{footline}%
    \insertshorttitle%
  \end{beamercolorbox}
  %
  \begin{beamercolorbox}[center,wd=.2\paperwidth,ht=2.5ex,dp=1.125ex,leftskip=.3cm,rightskip=.3cm plus1fil]{footline}%
    \centering
    \insertframenumber~of~\inserttotalframenumber%
  \end{beamercolorbox}%
  %
  \begin{beamercolorbox}[right,wd=.4\paperwidth,ht=2.5ex,dp=1.125ex,leftskip=.3cm,rightskip=.3cm plus1fil]{footline}%
    \hskip2ex plus1fill%
    \includegraphics[height=1.5ex]{cc}~%
    \includegraphics[height=1.5ex]{by}~~%
    % \includegraphics[height=1.5ex]{sa}~~%
    \insertshortauthor~%
  \end{beamercolorbox}%
}%
% -------------------------------
%%% 7.2.2 The Sidebars
% -------------------------------
%%% 7.2.3 Navigation Bars (funktioniert nur mit miniframe Themes)
% \setbeamertemplate{mini frames}[default] % shows small circles as mini frames.
% \setbeamertemplate{mini frames}[box] % shows small rectangles as mini frames.
% \setbeamertemplate{mini frames}[tick] % shows small vertical bars as mini frames.
% -------------------------------
%%% 7.2.4 The Navigation Symbols
%%% Beamer-Template/-Color/-Font navigation symbols
% \setbeamertemplate{navigation symbols}{} % suppresses all navigation symbols:
% \setbeamertemplate{navigation symbols}[horizontal] % Organizes the navigation symbols horizontally.
% \setbeamertemplate{navigation symbols}[vertical] % Organizes the navigation symbols vertically.
% \setbeamertemplate{navigation symbols}[only frame symbol] % Shows only the navigational symbol for navigating frames.
% -------------------------------
%%% 7.2.5 The Logo
% \setbeamertemplate{logo} % Beamer-Template/-Color/-Font
% -------------------------------
%%% 7.2.6 The Frame Title
% \setbeamertemplate{frametitle} % Beamer-Template/-Color/-Font
% \setbeamertemplate{frametitle}[default][center] % left, center, right
% \setbeamertemplate{frametitle}[shadow theme]
% \setbeamertemplate{frametitle}[sidebar theme]
% \setbeamertemplate{frametitle}[smoothbars theme]
% \setbeamertemplate{frametitle}[smoothtree theme]
% \setbeamertemplate{frametitle}
% {
%     \nointerlineskip
%     \begin{beamercolorbox}[sep=0.3cm,ht=2.5em,wd=\paperwidth]{frametitle}
%         \vbox{}\vskip-2ex%
%         \hspace{1.2cm}
%         \strut\insertframetitle\strut
%         \vskip-0.8ex%
%     \end{beamercolorbox}
% }
% % Minimalistic frametitle
% \setbeamertemplate{frametitle}
% {
%   \vspace{.2cm}
%   \hspace{-.8cm}
%   \insertframetitle
% }
% -------------------------------
%%% 7.2.7 The Background
% \setbeamertemplate{background canvas} % Beamer-Template/-Color/-Font
% \setbeamertemplate{background canvas}[default]
% \setbeamertemplate{background canvas}[vertical shading][ color options ] installs a vertically shaded background.
%     - top= color specifies the color at the top of the page. By default, 25% of the foreground of
%       the beamer-color palette primary is used.
%     - bottom= color specifies the color at the bottom of the page. By default, the background of
%       normal text at the moment of invocation of this command is used.
%     - middle= color specifies the color for the middle of the page. Thus, if this option is given, the
%       shading changes from the bottom color to this color and then to the top color.
%     - midpoint= factor specifies at which point of the page the middle color is used. A factor of 0
%       is the bottom of the page, a factor of 1 is the top. The default, which is 0.5 is in the middle.
% \setbeamertemplate{background canvas}[vertical shading][top=gray!40,bottom=gray!40,middle=white,midpoint=.9]
%\setbeamertemplate{background canvas}[vertical shading][top=white,bottom=gray!40]
% \setbeamercolor{background canvas}{bg=white}
%% Slide Background
\setbeamertemplate{background canvas}{%
	\begin{tikzpicture}
		\path [outer color = mypurple,
		       inner color = mypurple!75!white]
			(0,0) rectangle (\paperwidth,\paperheight);
	\end{tikzpicture}%
}

% \setbeamertemplate{background} % Beamer-Template/-Color/-Font
% \setbeamertemplate{background}[default] % is empty.
% \setbeamertemplate{background}[grid][step=1cm] % places a grid on the background.
%     - step= dimension specifies the distance between grid lines. The default is 0.5cm.
%     - color= color specifies the color of the grid lines. The default is 10% foreground.
% % Add global transition effect \transfade (breaks the background)
% \addtobeamertemplate{background canvas}{%
%   \transfade[duration=.1]%
% }{}
% -------------------------------
%%% 7.3 Margin Sizes
\setbeamersize{text margin left=2em,text margin right=2em}
% \setbeamersize{sidebar width left=2cm}
%         - text margin left= TEX dimension sets a new left margin. This excludes the left sidebar. Thus,
%           it is the distance between the right edge of the left sidebar and the left edge of the text.
%         - text margin right= TEX dimension sets a new right margin.
%         - sidebar width left= TEX dimension sets the size of the left sidebar. Currently, this command
%           should be given before a shading is installed for the sidebar canvas.
%         - sidebar width right= TEX dimension sets the size of the right sidebar.
%         - description width= TEX dimension sets the default width of description labels, see Section 11.1.
%         - description width of= text sets the default width of description labels to the width of the
%             text , see Section 11.1.
%         - mini frame size= TEX dimension sets the size of mini frames in a navigation bar. When two
%           mini frame icons are shown alongside each other, their left end points are TEX dimension far
%           apart.
%         - mini frame offset= TEX dimension set an additional vertical offset that is added to the mini
%           frame size when arranging mini frames vertically.
% -------------------------------
%%% 9.1 Adding a Title Page
% \setbeamersize{title page} % Beamer-Template/-Color/-Font
%    This template is invoked when the \titlepage command is used.
%    The following commands are useful for this template:
%     -  \insertauthor inserts a version of the author's name that is useful for the title page.
%     -  \insertdate inserts the date.
%     -  \insertinstitute inserts the institute.
%     -  \inserttitle inserts a version of the document title that is useful for the title page.
%     -  \insertsubtitle inserts a version of the document title that is useful for the title page.
%     -  \inserttitlegraphic inserts the title graphic into a template.
\defbeamertemplate*{title page}{flat2}[1][]
{
  \addtocounter{framenumber}{-1}
  \begin{center}
    \Huge{\inserttitle}\\[1ex]
    \hrule
    \vspace{1ex}
    \Large{\insertsubtitle}
    \vfill

    % \large\textcolor{normal text.fg}{\textit{\insertinstitute}, \insertdate}
    \vspace{1em}

    \vfill

    \large \insertauthor\\
    \vspace{-1ex}
    {\scriptsize \insertdate}\\
    \insertlogo
  \end{center}
}

% -------------------------------
%%% 9.2 Adding Sections and Subsections
% -------------------------------
%%% Parent Beamer-Template sections/subsections in toc
% This is a parent template, whose children are section in toc and subsection in toc.
% \setbeamertemplate{sections/subsections in toc}[default]
% \setbeamertemplate{sections/subsections in toc}[sections numbered]
% \setbeamertemplate{sections/subsections in toc}[subsections numbered]
% \setbeamertemplate{sections/subsections in toc}[circle]
% \setbeamertemplate{sections/subsections in toc}[square]
% \setbeamertemplate{sections/subsections in toc}[ball]
% \setbeamertemplate{sections/subsections in toc}[ball unnumbered]
% -------------------------------
%%% 9.6 Adding a Bibliography
% -------------------------------
% \setbeamertemplate{bibliography item} % Beamer-Template/-Color/-Font
% \setbeamertemplate{bibliography item}[default] %  little article icon as the reference
% \setbeamertemplate{bibliography item}[article] % Alias for the default.
% \setbeamertemplate{bibliography item}[book] % little book icon as the reference
% \setbeamertemplate{bibliography item}[triangle] % triangle as the reference
% \setbeamertemplate{bibliography item}[text] % reference text (like '[Dijkstra, 1982]')
% -------------------------------
%%% 10.1 Adding Hyperlinks and Buttons
% -------------------------------
%%% 11.1 Itemizations, Enumerations, and Descriptions
% \setbeamertemplate{items} % parent template of itemize items and enumerate items
% \setbeamertemplate{itemize items} % Parent Beamer-Template
% \setbeamertemplate{itemize items}[triangle]
% \setbeamertemplate{itemize items}[circle]
% \setbeamertemplate{itemize items}[square]
% \setbeamertemplate{itemize items}[ball]
% -------------------------------
% \setbeamertemplate{enumerate items}[default] % Numbered
% \setbeamertemplate{enumerate items}[circle] % Places the numbers inside little circles.
% \setbeamertemplate{enumerate items}[square] % Places the numbers on little squares.
% \setbeamertemplate{enumerate items}[ball] % 'Projects' the numbers onto little balls.
% -------------------------------
%%% 11.2 Hilighting
% -------------------------------
%%% 11.3 Block Environments
% \setbeamertemplate{blocks} % Parent Beamer-Template
% \setbeamertemplate{blocks}[default]
% \setbeamertemplate{blocks}[rounded][shadow=true]
 % \setbeamertemplate{blocks}[rounded][shadow=false]
% -------------------------------
%%% 11.4 Theorem Environments
% \setbeamertemplate{qed symbol} % Beamer-Template/-Color/-Font
% -------------------------------
% \setbeamertemplate{theorems} % Parent Beamer-Template
% \setbeamertemplate{theorems}[default]
% \setbeamertemplate{theorems}[normal font]
% \setbeamertemplate{theorems}[numbered]
% \setbeamertemplate{theorems}[ams style]
% -------------------------------
%%% 11.6 Figures and Tables
% \setbeamertemplate{caption} % Beamer-Template/-Color/-Font
% \setbeamertemplate{caption}[default] typesets the caption name (a word like 'Figure' or 'Abbildung' or 'Table')
% \setbeamertemplate{caption}[numbered] adds the figure or table number to the caption.
% \setbeamertemplate{caption}
% -------------------------------
% \setbeamertemplate{caption name} % Beamer-Color/-Font
% -------------------------------
%%% 11.10    Abstract
% -------------------------------
%%% 11.11 Verse, Quotations, Quotes
% -------------------------------
%%% 11.12 Footnotes
% -------------------------------
%%% 18.1 Specifying Note Contents
% \setbeamertemplate{note page} % Beamer-Template/-Color/-Font
% \setbeamertemplate{note page}[default]
% \setbeamertemplate{note page}[compress]
% \setbeamertemplate{note page}[plain]
% -------------------------------
%%% Specifying Which Notes and Frames Are Shown
 % \setbeameroption{hide notes}
% \setbeameroption{show notes}
% \setbeameroption{show notes on second screen= right }
% \setbeameroption{show only notes}


\renewcommand{\footnoterule}{}

\usepackage{hyperref}
\hypersetup{
  linktoc=all,
  bookmarks=false,           % show bookmarks bar?
  bookmarksopen,
  bookmarksnumbered,
  colorlinks = false,
  linkcolor=black,           % color of interlinks
  citecolor=black,           % color of the sitation links
  urlcolor=black,            % the url color
  unicode=true,              % non-Latin characters in Acrobat’s bookmarks
  pdftoolbar=true,           % show Acrobat’s toolbar?
  pdfmenubar=true,           % show Acrobat’s menu?
  pdffitwindow=true,         % window fit to page when opened
  pdftitle={Managed Runtime Systems},  % title
  pdfborder={ 0 0 0 },        % uBorder tin links
  pdfauthor = {Foivos Zakkak},
  pdfcreator = {Foivos Zakkak},
}

\usefonttheme{professionalfonts}% use own font handling
\usepackage{fontspec}
\usepackage{xunicode}
% \setmainfont{Comfortaa}
% \setsansfont{Comfortaa}
% \setsansfont{Liberation Sans}
\setmainfont[
% SmallCapsFont={Linux Biolinum},
SmallCapsFeatures={Letters=SmallCaps},
]{Liberation Sans}
% \setmonofont{Liberation Mono}
% \setmonofont{Inconsolata LGC for Powerline}
% Math fonts
\usepackage{unicode-math}
\setmathfont{xits-math.otf}

\usepackage{graphicx}
\usepackage{ulem}
\usepackage{color}
\usepackage{xspace}
\usepackage{xcolor}
\usepackage{array}
\usepackage{tikz}
\usetikzlibrary{arrows,arrows.meta,shapes,decorations,decorations.pathmorphing,calc,shadows,shadows.blur,shapes.multipart,positioning,patterns,fit,backgrounds,trees,shapes}

\tikzset{
    %Define standard arrow tip
    >=stealth',
    % Define arrow style
    arrow/.style={
           ->,
           thick,
           color=uompurple,
           shorten <=2pt,
           shorten >=4pt,},
    % Daniel's proposal for "uncovering" parts of a tikz-tree %
    invisible/.style={opacity=0},
    visible on/.style={alt=#1{}{invisible}},
    alt/.code args={<#1>#2#3}{%
      \alt<#1>{\pgfkeysalso{#2}}{\pgfkeysalso{#3}} % \pgfkeysalso doesn't change the path
    },
  }
\pgfdeclaredecoration{penciline}{initial}{
    \state{initial}[width=+\pgfdecoratedinputsegmentremainingdistance,
    auto corner on length=1mm,]{
        \pgfpathcurveto%
        {% From
            \pgfqpoint{\pgfdecoratedinputsegmentremainingdistance}
                      {\pgfdecorationsegmentamplitude}
        }
        {%  Control 1
        \pgfmathrand
        \pgfpointadd{\pgfqpoint{\pgfdecoratedinputsegmentremainingdistance}{0pt}}
                    {\pgfqpoint{-\pgfdecorationsegmentaspect
                     \pgfdecoratedinputsegmentremainingdistance}%
                               {\pgfmathresult\pgfdecorationsegmentamplitude}
                    }
        }
        {%TO
        \pgfpointadd{\pgfpointdecoratedinputsegmentlast}{\pgfpoint{1pt}{1pt}}
        }
    }
    \state{final}{}
  }
\usepackage{amsmath}
\usepackage{textpos}
% \usepackage{subfigure}

\graphicspath{{figures/}}
% \newcommand{\longname}{\emph{Source-Level Compiler Optimizations for Task-Parallelism}}
\newcommand{\java}[0]{Java\texttrademark\xspace}
\newcommand{\code}[1]{\texttt{#1}}

\usepackage{bbding} % for the Checkmark and XSolidBrush
\newcommand{\tik}[0]{{\color{YellowGreen}\Checkmark}} % Check-mark
\newcommand{\ex}[0]{{\color{BrickRed}\XSolidBrush}}  % X-mark

% \usepackage{verbatim}
\usepackage{listings}
\lstset{
  language=c,
  basicstyle=\small\ttfamily,
%  columns=flexible,
  % numbers=left,
  % numberstyle=\ttfamily\tiny,
  showstringspaces=false,
  alsoletter={-},
  literate={-}{-}1,
%  numbersep=1em,
  xleftmargin=2em,
%  xrightmargin=1em,
  escapeinside={@}{@},
%  morecomment=[l][\color{BrickRed}]{\#pragma\ task},
  commentstyle=\color{black},
  morekeywords={final, transient, synthetic},
%   identifierstyle=\color{green},
%  backgroundcolor=\color{Honeydew1},
  keywordstyle=\bfseries\color{myyellow},
  stringstyle=\color{YellowGreen},
%  moredelim=**[is][\only<+(1)->{\color{black}\lstset{style=highlight}}]{~}{~},
}

\usepackage{booktabs} % bottomrule
\usepackage{multirow}

% Print table of contects at the beggining of each section
% \AtBeginSection[]{\begin{frame}\frametitle{Table of Contents}\tableofcontents[currentsection,currentsubsection]\end{frame}}
% \AtBeginSubsection[]{\begin{frame}\frametitle{Table of Contents}\tableofcontents[currentsection,currentsubsection]\end{frame}}

% Add outlines
% \AtBeginSection[]
% {
%   {
%   \setbeamertemplate{footline}{}
%   \begin{frame}<beamer>[plain,noframenumbering]
%     \frametitle{Outline}
%     \tableofcontents[currentsection]
%   \end{frame}
%   }
% }
\setcounter{tocdepth}{1}

%%%%%%%%%%%%%%%
% Intro slide %
%%%%%%%%%%%%%%%

\title{Managed Runtime Systems}
\subtitle{Lecture 03: Memory Management}
\author[\url{https://foivos.zakkak.net}]{Foivos Zakkak}
% \institute[]{ManLang'17}
\date{\url{https://foivos.zakkak.net}}
% \logo{\includegraphics[height=1.0cm]{uniman-logo}}
%% Spread those bullets
% \let\olditem\item
% \renewcommand{\item}{\setlength{\itemsep}{\fill}\olditem}

\begin{document}
\setbeamercovered{invisible}

% \maketitle

\begin{frame}[plain]
  \titlepage
  \centering
  \includegraphics[height=.75cm]{cc}~
  \includegraphics[height=.75cm]{by}\\[1em]
  % \includegraphics[height=.75cm]{nc-eu}~
  % \includegraphics[height=.75cm]{sa}\\[1em]
  \scriptsize{Except where otherwise noted, this presentation is licensed under the\\
    \href{http://creativecommons.org/licenses/by/4.0/}%
    {Creative Commons Attribution 4.0 International License.}\\[1ex]
    Third party marks and brands are the property of their respective
    holders.}
\end{frame}

% \begin{frame}[plain,noframenumbering]{Outline}
%     \tableofcontents
% \end{frame}

\begin{frame}{Acknowledgments}
  The following slides are based on the corresponding slides of Mario Walczko about Memory Management:

  \begin{itemize}
  \item \href{https://www.dropbox.com/s/bnwq1q677spkglp/7\%20Memory\%20management.pdf}{Memory management part 1}
  \item \href{https://www.dropbox.com/s/gxsuu4uqbgwo88f/7B\%20Memory\%20Management\%2C\%20part\%202.pdf}{Memory management part 2}
  \item \href{https://www.dropbox.com/s/f0lwnc9zjtw8dxy/7C\%20Debugging\%20hints.pdf}{Memory management debugging hints}
\end{itemize}
\end{frame}

\section{Memory Management Introduction}

\begin{frame}{Memory Management Introduction}
  \centering
  Static vs Dynamic Allocation
\end{frame}

\begin{frame}{Static Memory Allocation}

  \begin{itemize}  \setlength{\itemsep}{\fill}
  \item Binding of name to memory addresses at \alert{compile/link} time
  \item All sizes are fixed, i.e. \alert{known at compile time}
  \item \alert{No stack} allocation
  \item Used in early FORTRAN, BASIC and various languages for \alert{embedded/real-time systems}
  \end{itemize}

\end{frame}

\begin{frame}{Static Memory Allocation}

  Pros \tik
  \begin{itemize}  \setlength{\itemsep}{\fill}
  \item \alert{No runtime overheads} for allocation/de-allocation
  \item Memory requirements \alert{known at compile time}
  \item \alert{No failures} due to lack of memory
  \end{itemize}

  \pause

  Cons \ex
  \begin{itemize}  \setlength{\itemsep}{\fill}
  \item Need to allocate and keep \alert{the maximum possible memory footprint} for the whole program execution
  \item \alert{No recursion} due to lack of stack allocation
  \end{itemize}

\end{frame}

\begin{frame}{Dynamic Allocation (on the stack)}
  \begin{itemize}  \setlength{\itemsep}{\fill}
  \item Languages (at least most of them) are \alert{based on procedures}
  \item \alert{LIFO/Depth-First} invocation order (in most cases\footnote{See Scheme and Smalltalk for counter-examples})
  \item Memory used by procedures can be \alert{managed as a stack}
  \item \alert{Hardware support} (SP register, call/ret instr.) since the 1960s
  \end{itemize}
\end{frame}

\begin{frame}{Dynamic Allocation (on the stack)}

  Pros \tik
  \begin{itemize}  \setlength{\itemsep}{\fill}
  \item \alert{Low runtime overheads}
  \item \alert{Bump stack pointer} for allocation
  \item \alert{Bulk} de-allocation on return
  \item No \alert{memory leaks}
  \end{itemize}

  \pause

  Cons \ex
  \begin{itemize}  \setlength{\itemsep}{\fill}
  \item Names passed as parameters, the deeper a procedure is invoked \alert{the more parameters it gets}
  \item \alert{Cannot return memory} to previous procedures
  \item Data-lifetime \alert{equals} the procedure's lifetime
  \item Can't handle \alert{complex data structures} like graphs
  \end{itemize}

\end{frame}

\begin{frame}{Dynamic Allocation (on the heap)}
  \begin{itemize}  \setlength{\itemsep}{\fill}
  \item \alert{Arbitrary requests} for memory segments
  \item Allocations \alert{may fail}
  \item Fast allocation vs Fast de-allocation \alert{trade-offs}
  \item Heap may \alert{not be contiguous}
  \end{itemize}
\end{frame}

\begin{frame}{Dynamic Allocation (on the heap)}
  Pros \tik
  \begin{itemize}  \setlength{\itemsep}{\fill}
  \item \alert{Arbitrary} allocation sizes
  \item Allocation and de-allocation \alert{from different procedures}
  \item Handling of \alert{complex data structures}
  \end{itemize}

  \pause

  Cons \ex
  \begin{itemize}  \setlength{\itemsep}{\fill}
  \item Noticeable \alert{runtime overheads}
  \item Need to perform \alert{a de-allocation for each allocation}\footnote{See region-based allocation for an enhancement}
  \item \alert{Memory leaks} are possible
  \end{itemize}
\end{frame}

\section{Heap Allocation Algorithms}

\begin{frame}{Heap Allocation}
  \centering
  \tikzstyle{allocated} = [rectangle, thick, draw, fill=myyellow, anchor=west, minimum width=1cm, minimum height=1cm]
  \tikzstyle{arrow} = [thick, ->]

  \begin{tikzpicture}
    \node[draw, minimum height=1cm, minimum width=10cm] (mem) {};
    \node[allocated] (alloc1) at (mem.west) {};
    \node[draw, above=of alloc1.east] (free) {Free};
    \draw[arrow, visible on=<1>] (free) to (alloc1.north east);
    \node[allocated, right=0cm of alloc1, minimum width=2cm, visible on=<2-5>] (alloc2) {};
    \draw[arrow, visible on=<2>] (free) to[bend left=15] (alloc2.north east);
    \node[allocated, right=0cm of alloc2, minimum width=1.5cm, visible on=<3->] (alloc3) {};
    \draw[arrow, visible on=<3>] (free) to[bend left=15] (alloc3.north east);
    \node[allocated, right=0cm of alloc3, minimum width=1.25cm, visible on=<4-5>] (alloc4) {};
    \draw[arrow, visible on=<4>] (free) to[bend left=15] (alloc4.north east);
    \node[allocated, right=0cm of alloc4, minimum width=2.2cm, visible on=<5->] (alloc5) {};
    \draw[arrow, visible on=<5->] (free) to[bend left=15] (alloc5.north east);
    \node[visible on=<6>] at (alloc2.base) {\textbf{?}};
    \node[visible on=<6>] at (alloc4.base) {\textbf{?}};
  \end{tikzpicture}
\end{frame}

\begin{frame}{Heap Allocation: Free-list variations }
  \begin{itemize}  \setlength{\itemsep}{\fill}
  \item \alert{First-Fit}:
    \begin{itemize}
    \item Search free-list from the beginning, peek first block that fits
    \item Add the remaining of the block (if any) as a new block to the list
    \item May result in a number of small blocks at beginning of free-list
    \end{itemize}
  \item \alert{Best-Fit}:
    \begin{itemize}
    \item Search free-list from the beginning, peek the block that best fits
    \item Add the remaining of the block (if any) as a new block to the list
    \item Reduces fragmentation
    \item Slow since we need to traverse the whole free-list
    \end{itemize}
  \item \alert{Next-Fit}:
    \begin{itemize}
    \item Search from where we stopped last time, peek the block that best fits
    \item Add the remaining of the block (if any) as a new block to the list
    \item Might increase fragmentation
    \item Often faster than First-fit
    \end{itemize}
  \end{itemize}
\end{frame}

\begin{frame}{First-Fit}
  \centering

  \tikzstyle{freeblock} = [draw, rectangle, anchor=north, thick, minimum width=2cm, minimum height=1cm]
  \tikzstyle{arrow} = [->, thick]
  \tikzstyle{arrowc} = [arrow, color=myyellow, dashed]

  \begin{tikzpicture}
    \node[freeblock, color=myyellow] (head) at (0,0) {Head};
    \node[freeblock, minimum height=2.5cm] (free1) at (2.5,0) {};
    \node[freeblock, minimum height=6cm, visible on=<-3>] (free2) at (5,0) {};
    \node[freeblock, minimum height=3cm, visible on=<4>] (free2b) at (5,0) {};
    \node[freeblock, minimum height=4cm] (free3) at (7.5,0) {};
    \node[freeblock, minimum height=5cm] (free4) at (10,0) {};

    \draw[arrowc] (head) -- (1.5,-0.5);
    \draw[arrow] (3.5,-0.5) -- (4,-0.5);
    \draw[arrow] (6,-0.5) -- (6.5,-0.5);
    \draw[arrow] (8.5,-0.5) -- (9,-0.5);

    \node[freeblock, minimum height=3cm, color=YellowGreen, visible on=<1>] (request) at (0,-3.25) {Request};
    \node[freeblock, minimum height=3cm, color=YellowGreen, visible on=<2>] (request) at (2.5,0) {Request};
    \node[freeblock, minimum height=3cm, color=YellowGreen, visible on=<3>] (request) at (5,0) {Request};
    \node[freeblock, minimum height=3cm, dashed, visible on=<4>] (request) at (5,-3.25) {Request};

  \end{tikzpicture}
\end{frame}

\begin{frame}{Best-Fit}
  \centering

  \tikzstyle{freeblock} = [draw, rectangle, anchor=north, thick, minimum width=2cm, minimum height=1cm]
  \tikzstyle{arrow} = [->, thick]
  \tikzstyle{arrowc} = [arrow, color=myyellow, dashed]

  \begin{tikzpicture}
    \node[freeblock, color=myyellow] (head) at (0,0) {Head};
    \node[freeblock, minimum height=2.5cm] (free1) at (2.5,0) {};
    \node[freeblock, minimum height=6cm] (free2) at (5,0) {};
    \node[freeblock, minimum height=4cm, visible on=<-6>] (free3) at (7.5,0) {};
    \node[freeblock, minimum height=1cm, visible on=<7>] (free3b) at (7.5,0) {};
    \node[freeblock, minimum height=5cm] (free4) at (10,0) {};

    \draw[arrowc] (head) -- (1.5,-0.5);
    \draw[arrow] (3.5,-0.5) -- (4,-0.5);
    \draw[arrow] (6,-0.5) -- (6.5,-0.5);
    \draw[arrow] (8.5,-0.5) -- (9,-0.5);

    \node[freeblock, minimum height=3cm, color=YellowGreen, visible on=<1>] (request) at (0,-3.25) {Request};
    \node[freeblock, minimum height=3cm, color=YellowGreen, visible on=<2>] (request) at (2.5,0) {Request};
    \node[freeblock, minimum height=3cm, color=YellowGreen, visible on=<3>] (request) at (5,0) {Request};
    \node[freeblock, minimum height=3cm, color=YellowGreen, visible on=<{4,6}>] (request) at (7.5,0) {Request};
    \node[freeblock, minimum height=3cm, color=YellowGreen, visible on=<{5}>] (request) at (10,0) {Request};

    \node[freeblock, minimum height=3cm, dashed, visible on=<7>] (request) at (7.5,-1.25) {Request};

  \end{tikzpicture}
\end{frame}

\begin{frame}{Next-Fit}
  \centering

  \tikzstyle{freeblock} = [draw, rectangle, anchor=north, thick, minimum width=2cm, minimum height=1cm]
  \tikzstyle{arrow} = [->, thick]
  \tikzstyle{arrowc} = [arrow, color=myyellow, dashed]

  \begin{tikzpicture}
    \node[freeblock, color=myyellow] (head) at (0,0) {Head};
    \node[freeblock, minimum height=2.5cm] (free1) at (2.5,0) {};
    \node[freeblock, minimum height=6cm, visible on=<-3>] (free2) at (5,0) {};
    \node[freeblock, minimum height=3cm, visible on=<4->] (free2b) at (5,0) {};
    \node[freeblock, minimum height=4cm] (free3) at (7.5,0) {};
    \node[freeblock, minimum height=5cm, visible on=<-8>] (free4) at (10,0) {};
    \node[freeblock, minimum height=.5cm, visible on=<9>] (free4b) at (10,0) {};

    \draw[arrowc, visible on=<-2>] (head) -- (1.5,-0.5);
    \draw[arrowc, visible on=<3-6>] (head.north east) to[bend left] (free2.north west);
    \draw[arrowc, visible on=<7>] (head.north east) to[bend left=10] (free3.north west);
    \draw[arrowc, visible on=<8->] (head.north east) to[bend left=10] (free4.north west);
    \draw[arrow] (3.5,-0.5) -- (4,-0.5);
    \draw[arrow] (6,-0.5) -- (6.5,-0.5);
    \draw[arrow] (8.5,-0.5) -- (9,-0.5);
    \draw[arrow] (free4.north west) to[bend right=15] (free1.north east);

    \node[freeblock, minimum height=3cm, color=YellowGreen, visible on=<1>] at (0,-3.25) {Request};
    \node[freeblock, minimum height=3cm, color=YellowGreen, visible on=<2>] at (2.5,0) {Request};
    \node[freeblock, minimum height=3cm, color=YellowGreen, visible on=<3>] at (5,0) {Request};
    \node[freeblock, minimum height=3cm, dashed, visible on=<4>] at (5,-3.25) {Request};

    \node[freeblock, minimum height=4.5cm, color=YellowGreen, visible on=<5>] at (0,-1.75) {Request};
    \node[freeblock, minimum height=4.5cm, color=YellowGreen, visible on=<6>] at (5,0) {Request};
    \node[freeblock, minimum height=4.5cm, color=YellowGreen, visible on=<7>] at (7.5,0) {Request};
    \node[freeblock, minimum height=4.5cm, color=YellowGreen, visible on=<8>] at (10,0) {Request};
    \node[freeblock, minimum height=4.5cm, dashed, visible on=<9>] at (10,-.75) {Request};

  \end{tikzpicture}
\end{frame}

\begin{frame}{Fragmentation}
  \begin{itemize}  \setlength{\itemsep}{\fill}
  \item Fragmentation is the phenomenon of \alert{not being able to use parts of memory} because of inefficient management
  \item A \alert{heavily fragmented} system may have \alert{plenty of free memory}, but chopped in small blocks that \alert{don't fit} a new request
  \end{itemize}

  \centering
  \tikzstyle{allocated} = [rectangle, thick, fill=myyellow, minimum width=1cm, minimum height=1cm]

  \begin{tikzpicture}
    \node[allocated] (alloc1) {};
    \node[allocated, right=.7cm of alloc1, minimum width=2cm] (alloc2) {};
    \node[allocated, right=.6cm of alloc2, minimum width=1.5cm] (alloc3) {};
    \node[allocated, right=.8cm of alloc3, minimum width=1.25cm] (alloc4) {};
    \node[allocated, right=.5cm of alloc4, minimum width=2.2cm] (alloc5) {};
    \node[allocated, right=.6cm of alloc5, minimum width=1.5cm] (alloc6) {};

    \node[fit=(alloc1)(alloc6), draw, inner sep=0mm] {};
  \end{tikzpicture}

\end{frame}

\begin{frame}{Fragmentation Categorization}
  \begin{itemize}  \setlength{\itemsep}{\fill}
  \item \alert{Internal Fragmentation} is the result of allocating larger chunks than actually needed (often due to alignment restrictions)
  \item \alert{External Fragmentation} is the result of constantly splitting free blocks, resulting in multiple small non-contiguous free blocks that cannot be used
  \end{itemize}
\end{frame}

% Todo add animations for the above algorithms

\begin{frame}{Free-list Allocation Main Overheads}
  \begin{enumerate}  \setlength{\itemsep}{\fill}
  \item Loop over free-blocks, even over obvious non-matches
  \item For each block check if it fits
  \item Split the block if it's bigger
  \end{enumerate}
\end{frame}

\begin{frame}{Single-size Free-lists}
  \begin{enumerate}  \setlength{\itemsep}{\fill}
  \item A set of free-lists, one for each size (for a set of common sizes)
  \item A generic free-list for the rest
  \item Always peek the first block from the list of the desired size
  \item If empty take a block from the generic one and split it
  \end{enumerate}
\end{frame}

\begin{frame}{Single-size Free-lists}
  \centering

  \tikzstyle{freeblock} = [draw, rectangle, anchor=north, thick, minimum width=2cm, minimum height=1cm]
  \tikzstyle{arrow} = [->, thick]
  \tikzstyle{arrowc} = [arrow, color=myyellow, dashed]

  \begin{tikzpicture}
    \node[freeblock, color=myyellow] (head1) at (0,0) {Head1};
    \node[freeblock, minimum height=1cm] at (2.5,0) {};
    \node[freeblock, minimum height=1cm] at (5,0) {};
    \node[freeblock, minimum height=1cm] at (7.5,0) {};
    \node[freeblock, minimum height=1cm] at (10,0) {};

    \draw[arrowc] (1,-0.5) -- (1.5,-0.5);
    \draw[arrow] (3.5,-0.5) -- (4,-0.5);
    \draw[arrow] (6,-0.5) -- (6.5,-0.5);
    \draw[arrow] (8.5,-0.5) -- (9,-0.5);

    \node[freeblock, color=myyellow] (head2) at (0,-1.5) {Head2};
    \node[freeblock, minimum height=1.5cm] at (2.5,-1.5) {};
    \node[freeblock, minimum height=1.5cm] at (5,-1.5) {};
    \node[freeblock, minimum height=1.5cm] at (7.5,-1.5) {};
    \node[freeblock, minimum height=1.5cm] at (10,-1.5) {};

    \draw[arrowc] (1,-2) -- (1.5,-2);
    \draw[arrow] (3.5,-2) -- (4,-2);
    \draw[arrow] (6,-2) -- (6.5,-2);
    \draw[arrow] (8.5,-2) -- (9,-2);

    \node[freeblock, color=myyellow] (head3) at (0,-3.5) {Head3};
    \node[freeblock, minimum height=2cm] at (2.5,-3.5) {};
    \node[freeblock, minimum height=2cm] at (5,-3.5) {};
    \node[freeblock, minimum height=2cm] at (7.5,-3.5) {};
    \node[freeblock, minimum height=2cm] at (10,-3.5) {};

    \draw[arrowc] (1,-4) -- (1.5,-4);
    \draw[arrow] (3.5,-4) -- (4,-4);
    \draw[arrow] (6,-4) -- (6.5,-4);
    \draw[arrow] (8.5,-4) -- (9,-4);

    \node[fit=(head1)(head3), draw=myyellow, dashed] (heads) {};
    \node[left=3mm of heads.north west, rotate=90, text=myyellow] {Heads};
  \end{tikzpicture}
\end{frame}

\begin{frame}
  \centering
  What about \alert{multi-threaded} applications???
\end{frame}

\section{Garbage Collection}

\begin{frame}{Garbage Collection}
  \begin{enumerate}  \setlength{\itemsep}{\fill}
  \item Ease programming, no need to argue about object lifetimes
  \item Eliminate errors due to dangling pointers
  \item Take care of the previous issues
  \item Still possible to leak memory!
  \end{enumerate}
\end{frame}

\begin{frame}[fragile]{How does it work?}
\begin{lstlisting}
public static void main(String args[]) {
  List<String> lines=
    Files.readAllLines(Paths.get(args[0]),
                       Charset.defaultCharset());
  int nLines= lines.size();
  // reclaim lines??
  System.out.println(nLines);
}
\end{lstlisting}
\end{frame}

\begin{frame}{Liveness}
  \begin{itemize}
  \item An object is \textit{dead} when it is \alert{no longer needed}
    \pause
    \begin{itemize}
    \item ``But, VMs (and compilers) have severely limited crystal balls''\\
      \hfill -- Mario Walczko
    \end{itemize}
  \end{itemize}
\end{frame}

\begin{frame}{Liveness in the Real World}
  \begin{itemize}
  \item An object is \textit{dead} when it is \alert{no longer reachable}
    \begin{itemize}  \setlength{\itemsep}{\fill}
    \item Reachable is an object that can be reached by following pointers starting from the system's roots
    \item The system's roots are all the variables in scope (of all threads)
    \item Requires traversal of stacks and globals
    \end{itemize}
  \end{itemize}
\end{frame}

\begin{frame}{Reachability}
  \centering

  \tikzstyle{object} = [draw, rectangle, thick, minimum width=1cm, minimum height=1cm]
  \tikzstyle{arrow} = [->, thick]
  \tikzstyle{arrowc} = [arrow, color=myyellow]

  \begin{tikzpicture}
    \node[object, color=myyellow, minimum width=1.5cm, minimum height=4cm] (roots) at (-2,-1) {roots};

    \node[object] (a) at (0,0) {A};
    \node[object, visible on=<-5>] (b) at (2,.5) {B};
    \node[object, visible on=<-5>] (c) at (4,0) {C};
    \node[object] (d) at (6,0) {D};
    \node[object] (e) at (8,0) {E};
    \node[object] (f) at (1,-2) {F};
    \node[object] (g) at (3,-3) {G};
    \node[object] (h) at (5,-2) {H};

    \draw[arrow] (a) to (f);
    \draw[arrow] (a) to (h);
    \draw[arrow] (b) to (c);
    \draw[arrow] (c) to (h);
    \draw[arrow] (h) to (d);
    \draw[arrow] (h) to (g);
    \draw[arrow] (d) to (e);
    \draw[arrow] (e) to (h);

    \draw[arrowc, visible on=<2->] (roots) to (a);
    \draw[arrowc, visible on=<2->] (roots) to (f);
    \draw[arrowc, visible on=<3->] (a) to (f);
    \draw[arrowc, visible on=<3->] (a) to (h);
    \draw[arrowc, visible on=<4->] (h) to (d);
    \draw[arrowc, visible on=<4->] (h) to (g);
    \draw[arrowc, visible on=<5->] (d) to (e);
    \draw[arrowc, visible on=<6->] (e) to (h);

    \node[object, draw=myyellow, visible on=<2->] at (0,0) {A};
    \node[object, draw=myyellow, visible on=<2->] at (1,-2) {F};
    \node[object, draw=myyellow, visible on=<3->] at (5,-2) {H};
    \node[object, draw=myyellow, visible on=<4->] at (6,0) {D};
    \node[object, draw=myyellow, visible on=<4->] at (3,-3) {G};
    \node[object, draw=myyellow, visible on=<5->] at (8,0) {E};
    \node[object, dashed, visible on=<6>] at (2,.5) {B};
    \node[object, dashed, visible on=<6>] at (4,0) {C};

  \end{tikzpicture}
\end{frame}

\subsection{Reference Counting}

\begin{frame}{Reference Counting}
  \begin{itemize}  \setlength{\itemsep}{\fill}
  \item Keep a reference \alert{counter per object}
  \item \alert{Increment} it when a reference to that object is assigned to a variable
  \item \alert{Decrement} it when a reference to that object is overwritten
  \item If the counter is \alert{zero}, the object \alert{can be reclaimed}
  \end{itemize}
\end{frame}

% todo draw example for reference counting

\begin{frame}{Reference Counting Drawbacks}
  \begin{enumerate}  \setlength{\itemsep}{\fill}
  \item Reference counting has to be \alert{performed on all variables}\\ (stack, global, and heap)
  \item References in an activation record have to be \alert{decremented before de-allocating the frame} upon return
  \item Decrementing the reference counter \alert{often incurs a cache miss}
  \item Decrementing the reference counter \alert{always incurs a write}
  \item Concurrent threads might \alert{contend on the reference counter}
  \item \alert{Cannot reclaim cycles}
  \end{enumerate}
\end{frame}

\begin{frame}{Reference Counting Delay Reclamation}
  \begin{itemize}  \setlength{\itemsep}{\fill}
  \item Avoid \alert{unbound recursion}
  \item \alert{Reduce the number of pauses} for Garbage Collection
  \end{itemize}
\end{frame}

\subsection{Tracing Collection}

\begin{frame}{Tracing Collection: Mark-n-Sweep}
  \begin{itemize}  \setlength{\itemsep}{\fill}
  \item Mark: Follow the system's roots and \alert{mark reachable objects}
  \item Sweep: \alert{Reclaim unmarked objects} at the end
  \end{itemize}
\end{frame}

\begin{frame}{Mark Implementations}
  \begin{itemize}  \setlength{\itemsep}{\fill}
  \item \alert{Recursive}: Worst case each object creates an activation (a single linked-list)
  \item \alert{Work queue}: Each object creates a new node in the queue
  \end{itemize}
\end{frame}

\begin{frame}{Sweep Implementation}
  \begin{itemize}  \setlength{\itemsep}{\fill}
  \item Add reclaimed chunks to a \alert{free-list}
  \item Requires \alert{parsing the whole heap} to find the non-marked objects
    \begin{itemize}
    \item  Possible (using the object headers), but \alert{inefficient}
    \end{itemize}
  \item \alert{Coalescing} adjacent free blocks
  \end{itemize}
\end{frame}

\begin{frame}{Compacting Sweep}
  \begin{itemize}  \setlength{\itemsep}{\fill}
  \item Move \textit{live} objects to consecutive memory addresses
    \begin{itemize}
    \item Moving objects \alert{breaks references} though
    \end{itemize}
  \item Create a \alert{contiguous large free space} after the \textit{live} objects
  \item Run on \alert{every collection} or when heap is \alert{heavily fragmented}
  \end{itemize}
\end{frame}

\begin{frame}{Compacting Sweep with Forwarding Pointers}
  \alert{Amend} each object with a forwarding pointer in its header\\

  \begin{enumerate}  \setlength{\itemsep}{\fill}
  \item \alert{Compute} forwarding pointers
  \item \alert{Update} all pointers using the forwarding pointers
  \item \alert{Move} the objects
  \end{enumerate}
\end{frame}

\begin{frame}{Compacting Sweep with Temporary Table}
  \begin{itemize}  \setlength{\itemsep}{\fill}
  \item Instead of using a forwarding pointer \alert{replace actual header with pointer to a temporary table entry}
  \item Each temporary table entry holds \alert{a header and a forwarding location}
  \end{itemize}
\end{frame}

\begin{frame}{Compacting Sweep with \textit{Threading}}
  \begin{enumerate}  \setlength{\itemsep}{\fill}
  \item Replace the object header with a pointer to a list
  \item This starts from the object and goes through all the fields that reference it
  \item The last field in the list contains the initial content of the object header
  \item When the object is moved the list is traversed to update the corresponding fields
  \end{enumerate}
\end{frame}

\subsection{Copying Collection}

\begin{frame}{Copying Collection}
  \begin{itemize}  \setlength{\itemsep}{\fill}
  \item \alert{Trace and compaction} combination
  \item \alert{Split memory} in \textit{from} and \textit{to} semi-spaces
  \item \alert{Copy} \textit{live} object on trace to \textit{to} semi-space
  \item \alert{Leave} forwarding pointers in \textit{from} semi-space
  \item At the end, \textit{from} becomes \textit{to} and vice versa
  \end{itemize}
\end{frame}

\begin{frame}{Copying Collection}
  Pros \tik
  \begin{itemize}  \setlength{\itemsep}{\fill}
  \item \alert{Bump} allocation
  \item Traverse \alert{only live} objects
  \item \alert{Increase locality?}
  \end{itemize}

  \pause

  Cons \ex
  \begin{itemize}  \setlength{\itemsep}{\fill}
  \item Requires \alert{twice the memory}
  \item \alert{Copies the whole heap} in each collection
  \end{itemize}
\end{frame}

% Closing slide
% \subsection{Closure}
% % {
% % \setbeamertemplate{footline}{}
% % \begin{frame}[noframenumbering]
% %  \frametitle{\fontspec{Purisa}\textbf{Thank You!}}
% %  \centering
% %  \titlepage
% % \end{frame}
% % }
% \begin{frame}
%   % \centering
%   \Huge
%   \alert{Can managed programming languages run on future
%   many-core architectures?}
%   % \vspace{-2em}
%   \vfill
%   \begin{flushright}
%     \fontspec{Purisa}\structure{Thank You!}
%   \end{flushright}
% \end{frame}

%   BACKUP SLIDES

\end{document}
